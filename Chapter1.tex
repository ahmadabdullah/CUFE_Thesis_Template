Usually, the first chapter of the thesis provides an introduction to the research work. Each chapter may start with an introductory paragraph right after its title to provide some information about its content.

\section{First section}

Use body text style (FECU Thesis Body Text) for writing text throughout the thesis.
Body text. Body text. Body text. Body text. Body text. Body text. Body text. Body text. Body text. Body text. Body text. Body text. Body text. Body text. Body text. Body text. Body text. Body text. Body text. Body text. Body text. Body text. Body text. Body text. Body text. Body text. Body text. Body text. Body text.
Body text. Body text. Body text. Body text. Body text. Body text. Body text. Body text. Body text. Body text. Body text. Body text. Body text. Body text. Body text. Body text. Body text. Body text. Body text. Body text. Body text. Body text. Body text. Body text. Body text. Body text. Body text. Body text. Body text. Body text. Body text. Body text. Body text. Body text. Body text. Body text. Body text. Body text. Body text. Body text. Body text. Body text. Body text.

\section{Second section}

There are three levels of headings in this template. Using the heading styles allows for automatic numbering of all sections and helps in automatically generating the table of contents. A new abbreviation or symbol \newnom{Description}{Symbol} is defined using the \verb+\newnom{Description}{Symbol}+ command.

\section{Heading level 1}

\subsection{Heading level 2}

\subsubsection{Heading level 3}

\section{Organization of the thesis}

The remainder of this thesis organized as follows. \chref{‎Chapter2} provides a detailed survey of the previous studies.

